
\begin{question}
To determine kinetic data, the reaction rate of chemical reaction is determined
\renewcommand{\labelenumi}{(\alph{enumi})}
\begin{enumerate}
\item What is the dimension of the reaction rate?
\item What are the dimension of the rate constant?
\end{enumerate}
\end{question}

\begin{solution}
The general rate law can be formulated by:
%%
\begin{equation}
 r = \frac{1}{\nu_i} * \frac{\mathrm{d}c_i}{\mathrm{d}t} = k * c_i^n \qquad \text{with } [r] = \si{\mole\per\liter\per\second}
\end{equation}
%
\begin{table}[H]
 \centering
 \caption{Rate law and dimension of $k$}
 \begin{tabular}{l l l}
 \toprule
  Order & Rate law & $[k]$ \\
  \midrule
  0 & $r = k$               & \si{\mole\per\liter\per\second} \\
  1 & $r = k*c_A $          & \si{\liter\per\second\per\mole} \\
  2 & $r = k*c_A*c_B $      & \si{\square\liter\per\square\second\per\square\mole} \\
  3 & $r = k*c_A*c_B*c_C $  & \si{\cubic\liter\per\cubic\second\per\cubic\mole} \\
  \bottomrule
 \end{tabular}
\end{table}
%%
The rate constant of a reaction with oder $n \neq 0$ has the dimension $[k] = \si{\liter\tothe{n}\second\tothe{-n}\mole\tothe{-n}}$
\end{solution}
