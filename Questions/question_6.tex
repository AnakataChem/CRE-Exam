\begin{question}
A heterogeneously catalysed reaction of two gaseous reactants (\ch{A + B}) to the product C shows \textbf{Langmuir-Hinshelwood} kinetics. In an experiment, the partial pressure of component B is kept constant, while that of component A is steadily increased. Describe the course of the reaction rate and explain it using a suitable formula.  For this purpose, assign the areas of the curve exactly to the explained cases (in the diagram)! 

\end{question}
\begin{solution}
The Langmuir-Hinshelwood model states that both reactants are chemisorpt on the surface of the catalyst.
%%
\begin{equation*}
\ch{A_{ch} + B_{ch} -> C}
\end{equation*}
%%
The reaction rate can be described by:
%%
\begin{equation}
r = k * \Theta_A * \Theta_B
\end{equation}
%%
The coverage $\Theta$ can be calculated by:
%%
\begin{equation}
\Theta_A = \frac{K_A * p_A}{1 + K_A * p_A + K_B * p_B} \qquad\qquad \Theta_B = \frac{K_B * p_B}{1 + K_A * p_A + K_B * p_B} 
\end{equation}
%%
Thus the reaction rate can be rearranged to:
%%
\begin{equation}\label{eqn6:rate}
r = k * \frac{K_A * p_A * K_B * p_B}{(1 + K_A * p_A + K_B * p_B)^2}
\end{equation}
%%
In the case of a low partial pressure of A $(K_A * p_A \ll 1 + K_B * p_B)$, Eq. \ref{eqn6:rate} can be simplified to:
%%
\begin{equation}
r = k *\frac{K_A * K_B * p_B}{(1 + K_B * p_B)^2}*p_A = k' * p_A 
\end{equation}
%%
The rate behaves 1st order in $p_A$. In the case of a heigh partial pressure of A $(K_A * p_A \gg 1 + K_B * p_B)$, Eq. \ref{eqn6:rate} can be simplified to:
%%
\begin{equation}
r = k * \frac{K_B * p_B}{K_A} * \frac{1}{p_A}
\end{equation}
%%
The rate behaves -1st order in $p_A$.
%%
\begin{center}
% This file was created by tikzplotlib v0.9.8.
\begin{tikzpicture}

\definecolor{color0}{rgb}{0.12156862745098,0.466666666666667,0.705882352941177}

\begin{axis}[
height=9cm,
ylabel near ticks,
xlabel near ticks,
xtick=\empty,
ytick=\empty,
width=10cm,
xlabel={\(\displaystyle p_A\)},
xmin=0, xmax=1.05,
ylabel={\(\displaystyle r\)},
ymin=0, ymax=1.5
]
\addplot [semithick, color0]
table {%
0 0
0.0101010101010101 0.120119391395084
0.0202020202020202 0.228827662721893
0.0303030303030303 0.327316008728427
0.0404040404040404 0.416631596666947
0.0505050505050505 0.497697520561443
0.0606060606060606 0.571329639889197
0.0707070707070707 0.638250842712152
0.0808080808080808 0.699103170680037
0.0909090909090909 0.754458161865569
0.101010101010101 0.80482570239334
0.111111111111111 0.850661625708885
0.121212121212121 0.892374256354786
0.131313131313131 0.930330061154564
0.141414141414141 0.964858543105369
0.151515151515152 0.996256490761985
0.161616161616162 1.02479167744941
0.171717171717172 1.05070608947546
0.181818181818182 1.07421875
0.191919191919192 1.09552819485375
0.202020202020202 1.11481464798883
0.212121212121212 1.13224193706499
0.222222222222222 1.14795918367347
0.232323232323232 1.16210229766559
0.242424242424242 1.1747953008188
0.252525252525253 1.18615150150006
0.262626262626263 1.1962745389649
0.272727272727273 1.20525931336742
0.282828282828283 1.21319281537761
0.292929292929293 1.22015486744469
0.303030303030303 1.22621878715815
0.313131313131313 1.23145198179907
0.323232323232323 1.23591648200743
0.333333333333333 1.2396694214876
0.343434343434343 1.24276346880907
0.353535353535354 1.24524721661192
0.363636363636364 1.24716553287982
0.373737373737374 1.24855987838216
0.383838383838384 1.24946859389946
0.393939393939394 1.24992716042189
0.404040404040404 1.24996843514053
0.414141414141414 1.24962286572788
0.424242424242424 1.24891868512111
0.434343434343434 1.24788208877346
0.444444444444444 1.24653739612188
0.454545454545455 1.24490719782707
0.464646464646465 1.24301249017381
0.474747474747475 1.24087279787081
0.484848484848485 1.23850628635767
0.494949494949495 1.23592986461078
0.505050505050505 1.23315927933673
0.515151515151515 1.23020920135082
0.525252525252525 1.22709330485689
0.535353535353535 1.22382434027308
0.545454545454546 1.22041420118343
0.555555555555556 1.21687398593835
0.565656565656566 1.2132140543758
0.575757575757576 1.2094440800895
0.585858585858586 1.2055730986294
0.595959595959596 1.20160955198334
0.606060606060606 1.19756132965597
0.616161616161616 1.19343580663138
0.626262626262626 1.18923987847976
0.636363636363636 1.18497999384426
0.646464646464647 1.18066218452319
0.656565656565657 1.17629209334294
0.666666666666667 1.171875
0.676767676767677 1.16741584503448
0.686868686868687 1.1629192520833
0.696969696969697 1.15838954854858
0.707070707070707 1.15383078480473
0.717171717171717 1.14924675205766
0.727272727272727 1.14464099895942
0.737373737373737 1.14001684707337
0.747474747474748 1.13537740527673
0.757575757575758 1.13072558318029
0.767676767676768 1.12606410363861
0.777777777777778 1.12139551441794
0.787878787878788 1.11672219908372
0.797979797979798 1.11204638716447
0.808080808080808 1.1073701636447
0.818181818181818 1.10269547783471
0.828282828282828 1.09802415166205
0.838383838383838 1.09335788742552
0.848484848484849 1.08869827504949
0.858585858585859 1.08404679887357
0.868686868686869 1.0794048440099
0.878787878787879 1.07477370229779
0.888888888888889 1.07015457788347
0.898989898989899 1.06554859245034
0.909090909090909 1.06095679012346
0.919191919191919 1.05638014207017
0.929292929292929 1.05181955081716
0.939393939393939 1.04727585430274
0.94949494949495 1.04274982968195
0.95959595959596 1.03824219690062
0.96969696969697 1.03375362205341
0.97979797979798 1.02928472054003
0.98989898989899 1.02483606003245
1 1.02040816326531
};
\draw (axis cs:0.1,0.5) node[
  anchor=base west,
  text=black,
  rotate=70.0
]{1st order};
\draw (axis cs:0.3,1.275) node[
  anchor=base west,
  text=black,
  rotate=0.0
]{zero order};
\draw (axis cs:0.7,1.175) node[
  anchor=base west,
  text=black,
  rotate=345.0
]{-1st order};
\end{axis}

\end{tikzpicture}

\end{center}
\end{solution}
