\begin{question}
For a zero order reaction, the following applies: $X = Da$
%%
\renewcommand{\labelenumi}{\alph{enumi})}
\begin{enumerate}
 \item What conversion do you expect with $Da = 2$
 \item Is the Damköhler number at this reaction order depended on the temperature?
 \item How do you explain this? Use equations for the explanation!
\end{enumerate}
\end{question}
%%%%%%%%%%%%%%%%%%%%%%%%%%%% Start Python calculations %%%%%%%%%%%%%%%%%%%%%%%%%%%%
\begin{pycode}

\end{pycode}
%%%%%%%%%%%%%%%%%%%%%%%%%%%% End Python calculations %%%%%%%%%%%%%%%%%%%%%%%%%%%%
\begin{solution}
The conversion can only be \SI{100}{\percent} even if $Da = 2$. The higher Damköhler number can be attributed to the temperature.
%%
\begin{equation}
Da = \frac{-\nu_A * r_0 * \tau}{c_{A0}}
\end{equation}
The reaction rate $r_0$ at the begin of the reaction $(X = 0)$:
%%
\begin{equation}
r_0 = k(T) = k_{\infty} * \exp\left(\frac{-E_A}{R * T}\right)
\end{equation}
\end{solution}
