\begin{question}
A heterogeneously catalysed reaction of two gaseous reactants (\ch{A + B}) to the product C shows \textbf{Eley-Rideal} kinetics. In an experiment, the partial pressure of component B is kept constant, while that of component A is steadily increased. Describe the course of the reaction rate and explain it using a suitable formula.  For this purpose, assign the areas of the curve exactly to the explained cases (in the diagram)! 

\end{question}
\begin{solution}
The Eley-Rideal model states that one reactant is chemisorpt on the surface of the catalyst.
%%
\begin{equation*}
\ch{A_{ch} + B_{g} -> C}
\end{equation*}
%%
The reaction rate can be described by:
%%
\begin{equation}
r = k * \Theta_A * p_B
\end{equation}
%%
The coverage $\Theta$ can be calculated by:
%%
\begin{equation}
\Theta_A = \frac{K_A * p_A}{1 + K_A * p_A}
\end{equation}
%%
Thus the reaction rate can be rearranged to:
%%
\begin{equation}\label{eqn14:rate}
r = k * \frac{K_A * p_A * p_B}{1 + K_A * p_A}
\end{equation}
%%
In the case of a low partial pressure of A $(K_A * p_A \ll 1)$, Eq. \ref{eqn14:rate} can be simplified to:
%%
\begin{equation}
r = k * K_A *  p_A * p_B
\end{equation}
%%
The reaction is 2nd order and the rate behaves 1st order in $p_A$. 

In the case of a high partial pressure of A $(K_A * p_A \gg 1)$, Eq. \ref{eqn14:rate} can be simplified to:
%%
\begin{equation}
r = k * p_B 
\end{equation}
%%
The reaction is 1st order and the rate behaves zero order in $p_A$. 
%%
\begin{center}
% This file was created by tikzplotlib v0.9.8.
\begin{tikzpicture}

\definecolor{color0}{rgb}{0.12156862745098,0.466666666666667,0.705882352941177}

\begin{axis}[
height=9cm,
width=10cm,
ylabel near ticks,
xlabel near ticks,
xtick=\empty,
ytick=\empty,
xlabel={$p_A$},
xmin=0, xmax=1.05,
ylabel={$r$},
ymin=0, ymax=1.5
]
\addplot [semithick, color0]
table {%
0 0
0.0101010101010101 0.0960760977047176
0.0202020202020202 0.182921578859269
0.0303030303030303 0.261423285081202
0.0404040404040404 0.33238285370617
0.0505050505050505 0.39652490388284
0.0606060606060606 0.454504436179757
0.0707070707070707 0.50691352126684
0.0808080808080808 0.554287345974482
0.0909090909090909 0.597109678470867
0.101010101010101 0.635817808366385
0.111111111111111 0.670807012192094
0.121212121212121 0.702434589852435
0.131313131313131 0.731023513271317
0.141414141414141 0.756865724490533
0.151515151515152 0.780225116899746
0.161616161616162 0.801340230041529
0.171717171717172 0.820426685510056
0.181818181818182 0.837679388818152
0.191919191919192 0.853274519717549
0.202020202020202 0.867371331296937
0.212121212121212 0.880113776229699
0.222222222222222 0.891631976778104
0.232323232323232 0.902043553565238
0.242424242424242 0.911454826683711
0.252525252525253 0.919961901406528
0.262626262626263 0.927651649587094
0.272727272727273 0.93460259677014
0.282828282828283 0.940885724072487
0.292929292929293 0.946565193022242
0.303030303030303 0.95169900075827
0.313131313131313 0.956339572280654
0.323232323232323 0.960534295800047
0.333333333333333 0.964326006652748
0.343434343434343 0.967753424723096
0.353535353535354 0.970851549840042
0.363636363636364 0.973652019185551
0.373737373737374 0.976183430364602
0.383838383838384 0.978471633435884
0.393939393939394 0.980539994885321
0.404040404040404 0.982409636238053
0.414141414141414 0.984099649745507
0.424242424242424 0.985627293350097
0.434343434343434 0.987008166918475
0.444444444444444 0.988256371542979
0.454545454545455 0.989384653538023
0.464646464646465 0.990404534601874
0.474747474747475 0.991326429472986
0.484848484848485 0.992159752282388
0.494949494949495 0.992913012688135
0.505050505050505 0.993593902773542
0.515151515151515 0.994209375596577
0.525252525252525 0.994765716192531
0.535353535353535 0.995268605755032
0.545454545454546 0.995723179650791
0.555555555555556 0.996134079860527
0.565656565656566 0.996505502381566
0.575757575757576 0.996841240076183
0.585858585858586 0.99714472140325
0.595959595959596 0.997419045428685
0.606060606060606 0.99766701347225
0.616161616161616 0.997891157713834
0.626262626262626 0.998093767051364
0.636363636363636 0.998276910474385
0.646464646464647 0.998442458192002
0.656565656565657 0.998592100730926
0.666666666666667 0.99872736619866
0.676767676767677 0.9988496358881
0.686868686868687 0.998960158382911
0.696969696969697 0.999060062307712
0.707070707070707 0.999150367853272
0.717171717171717 0.999231997194414
0.727272727272727 0.999305783907001
0.737373737373737 0.999372481480181
0.747474747474748 0.999432771010802
0.757575757575758 0.999487268158589
0.767676767676768 0.999536529433081
0.777777777777778 0.999581057876552
0.787878787878788 0.999621308200937
0.797979797979798 0.999657691431223
0.808080808080808 0.999690579102722
0.818181818181818 0.999720307055081
0.828282828282828 0.999747178861784
0.838383838383838 0.999771468930161
0.848484848484849 0.999793425303556
0.858585858585859 0.999813272194275
0.868686868686869 0.999831212273182
0.878787878787879 0.999847428739315
0.888888888888889 0.999862087190663
0.898989898989899 0.999875337315208
0.909090909090909 0.999887314419492
0.919191919191919 0.999898140810335
0.929292929292929 0.999907927043793
0.939393939393939 0.99991677305413
0.94949494949495 0.999924769174313
0.95959595959596 0.999931997058472
0.96969696969697 0.999938530515726
0.97979797979798 0.999944436263903
0.98989898989899 0.999949774610841
1 0.999954600070238
};
\draw (axis cs:0.08,0.2) node[
  anchor=base west,
  text=black,
  rotate=70
]{1st order};
\draw (axis cs:0.75,1.05) node[
  anchor=base west,
  text=black,
  rotate=0.0
]{zero order};
\end{axis}

\end{tikzpicture}

\end{center}
\end{solution}
