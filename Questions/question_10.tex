
\begin{question}
What is meant by stationarity in a reactor system? How does this affect the mass and heat balance?
\end{question}

\begin{solution}
A \textbf{stationary process} describes the movement of matter or energy in which the state variables of the considered system do not change in the course of time.
%%
\begin{itemize}
 \item In terms of the \textbf{mass balance}, an intermediate product of a reaction chain has a constant concentration over time during the process.
 \item With regard to the \textbf{heat balance}, the continuous heat transport is stationary if the temperature at any point in the system is constant in time.
\end{itemize}
%%
This implies mathematically for the mass balance:
%%
\begin{equation}
 \frac{\mathrm{d}c_A}{\mathrm{d}t} = - \nabla * \vec{j} + \nu_A * r \overset{!}{=} 0
\end{equation}
%%
And for the heat balance:
%%
\begin{equation}
 \frac{\mathrm{d}(\rho * c_p * T)}{\mathrm{d}t} =  - \nabla * \vec{\dot{q}} + (-\Delta H)* r \overset{!}{=} 0 
\end{equation}
%%
\end{solution}
