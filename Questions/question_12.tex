
\begin{question}
Radical polymerization is a system of consecutive reactions. Intermediate products are radicals.

\renewcommand{\labelenumi}{(\alph{enumi})}
\begin{enumerate}
 \item Explain the Bodenstein quasi-staionarity principle in context with radical polymerization.
 \item What is the degree of polymerization $P_n$. What measures are necessary in the process of radical polymerization to achieve the highest possible degree of polymerization $P_n$. Neglect the gel effect.
 \item What is the meaning of the radical efficiency factor?
\end{enumerate}
\end{question}

\begin{solution}
\textbf{Bodenstein's quasi-stationary principle} states that if an intermediate product (B) is formed slowly with rate constant $k_1$ and the final product (C) is formed quickly with rate constant $k_2$, it can be assumed that the concentration of the intermediate product does not change.

\begin{equation}
 \ch{A ->[ $k_1$ ] B ->[ $k_2$ ] C} \qquad \text{with } k_2 \gg k_1 
\end{equation}
%%
The \textbf{degree of polymerization} is the ratio of the number average of the molar mass of the macro molecule $M_n$ and the molar mass of the basic building block $M_m$ (monomer).
%%
\begin{equation}
 P_n = \frac{M_n}{M_m}
\end{equation}
%%
In case of a \textbf{radical polymerization} the degree of polymerization is proportional to the ratio of the rate of the propagation and the rate of the termination.
%%
\begin{equation}
 P_n \propto \frac{k_w}{k_a}
\end{equation}
%
A high degree of polymerization requires $k_w \gg k_a$.

The \textbf{radical efficiency factor} $f$ describes the effective radical concentration. The efficiency is ideal 1 but can be decreased due to recombination of the radical inside a solvent cage or before initiating a chain, or due to side reactions. 
\end{solution}
